
\documentclass[12pt,a4paper,titlepage]{article}

\usepackage[utf8]{inputenc}
\usepackage[english]{babel}
\usepackage[left=1.5cm,right=2cm,top=2cm,bottom=2cm]{geometry}


\usepackage{amsmath, amsxtra, amsfonts, amssymb, amstext, amsthm}

\usepackage{booktabs}
\usepackage{fullpage}
\usepackage{nicefrac}
\usepackage{xspace}
\usepackage[noadjust]{cite}
\usepackage{url}\urlstyle{rm}
\usepackage{graphicx}
% \usepackage{graphics}
\usepackage[space]{grffile} 

\usepackage[usenames,dvipsnames]{xcolor}
\usepackage[colorlinks]{hyperref}
\definecolor{linkblue}{rgb}{0.1,0.1,0.8}
\hypersetup{colorlinks=true,linkcolor=linkblue,filecolor=linkblue,urlcolor=linkblue,citecolor=linkblue}
% \usepackage[algo2e,ruled,vlined]{algorithm2e}


\usepackage{wrapfig}
\usepackage{pdfpages}




% --- Code part
\usepackage{listings}
\lstset{language=Python} % Changer eventuellement le nom du langage
\newcommand{\class}[1]{\texttt{#1}}

% --- Math part
\newtheorem{theorem}{Theorem}
\newtheorem{lemma}[theorem]{Lemma}
\newtheorem{proposition}[theorem]{Proposition}
\newtheorem{corollary}[theorem]{Corollary}
\newtheorem{definition}[theorem]{Definition}
\newtheorem{algorithm}[theorem]{Algorithm}
\newtheorem{remark}[theorem]{Remark}



% Mathematic abbreviations

\newcommand{\N}{\mathbb{N}}
\newcommand{\R}{\mathbb{R}}
\newcommand{\Z}{\mathbb{Z}}

\newcommand{\Esp}{\mathbb{E}}
\newcommand{\Prob}{\mathbb{P}}
\newcommand{\Var}{\text{Var}}

\newcommand{\A}{\frak{A}}
\newcommand{\F}{\mathbb{F}}


\renewcommand{\epsilon}{\varepsilon}


% More definitions
\usepackage{showkeys}
\usepackage{mathtools}
\usepackage{multirow}
\usepackage{appendix}



\author{Sylvain Colin \& Gaspard Férey}
\title{The Sidelnikov-Shestakov's Attack applied to the Chor-Rivest Cryptosystem}

\begin{document}

\maketitle
\tableofcontents
\newpage




\begin{abstract}

In this article, we discuss about the Sidelnikov-Shestakov Attack on cryptosystems based on Reed-Solomon codes. Then we describe how this algorithm can be used to attack the Chor-Rivest Cryptosystem.



\end{abstract}


\section{Introduction}
\label{sec:intro}


\subsection{Our Work}


\section{Preliminaries}
\label{sec:defnot}


\subsection{A cryptosystem based on Reed-Solomon codes}

We study here the public-key cryptosystem introduced by Niederreiter \cite{NiederH86} applied to the generalized Reed-Solomon codes. Let $\F_q$ be a finite field with $q = p^h$ elements and $\F = \F_q \cup \{ \infty \}$, where $\infty$ has natural properties ( $1/\infty = 0$, etc). We call $\A$ the following matrix:
$$ \A(\alpha_1, \ ... \ , \alpha_n, z_1, \ ... \ , z_n) := \left(
\begin{array}{cccc}
z_1 \alpha_1^0 &  z_2 \alpha_2^0 & \cdots & z_n \alpha_n^0 \\
z_1 \alpha_1^1 &  z_2 \alpha_2^1 & \cdots & z_n \alpha_n^1 \\
 & & \ddots & \\
z_1 \alpha_1^{k-1} &  z_2 \alpha_2^{k-1} & \cdots & z_n \alpha_n^{k-1}
\end{array}
\right) \in \mathcal{M}_{\F_q}(k,n) $$



\subsection{Equivalence between Reed-Solomon codes}

Sidelnikov and Shestakov show \cite{SidelShes92} that for all $a \in \F_q-\{0\}$ and $b \in \F_q$, there exists $H_1, H_2, H_3 \in \mathcal{M}_{F_q}(k,k)$ invertible such that
\begin{eqnarray*}
H_1 \A(a\cdot\alpha_1 + b, \ ... \ , a\cdot \alpha_n + b, c_1z_1, \ ... \ , c_n z_n) &=& \A(\alpha_1, \ ... \ , \alpha_n, z_1, \ ... \ , z_n) \\
H_2 \A\left( \frac{1}{\alpha_1}, \ ... \ , \frac{1}{\alpha_n}, d_1z_1, \ ... \ , d_n z_n \right) &=& \A(\alpha_1, \ ... \ , \alpha_n, z_1, \ ... \ , z_n) \\
H_3 \A\left( \alpha_1, \ ... \ , \alpha_n, a\cdot z_1, \ ... \ , a\cdot z_n \right) &=& \A(\alpha_1, \ ... \ , \alpha_n, z_1, \ ... \ , z_n)
\end{eqnarray*}
This means that for any cryptosystem $M = H \A(\alpha_1, \ ... \ , \alpha_n, z_1, \ ... \ , z_n)$, for any birationnal transformation
$$ \phi : x \mapsto \frac{ax+b}{cx+d} $$
$M = H_{\phi} \A(\phi(\alpha_1), \ ... \ , \phi(\alpha_n), z'_1, \ ... \ , z'_n)$ and by using the unique transformation $\phi$ that maps $(\alpha_1, \alpha_2, \alpha_3)$ to $(0,1,\infty)$, we get that for any cryptosystem $M = H \A(\alpha_1, \ ... \ , \alpha_n, z_1, \ ... \ , z_n)$, $M$ can be uniquely written
$$ M = H' \A(0, 1, \infty, \alpha'_4, \ ... \ , \alpha'_n, 1, z'_2, \ ... \ , z'_n) $$
with $H'$ invertible, $z'_i \neq 0$ and $\alpha_i$ distincts elements of $\F_q-\{0, 1, \infty)$.

\section{Attack of Sidelnikov-Shestakov}




\section{Application to the Chor-Rivest Cryptosystem}



\section{Conclusions}


\bibliographystyle{abbrv}
\bibliography{references}
\newpage
\appendix
\end{document}

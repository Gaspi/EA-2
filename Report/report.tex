
\documentclass[12pt,a4paper,titlepage]{article}

\usepackage[utf8]{inputenc}
\usepackage[english]{babel}
\usepackage[left=1.5cm,right=2cm,top=2cm,bottom=2cm]{geometry}


\usepackage{amsmath, amsxtra, amsfonts, amssymb, amstext, amsthm}

\usepackage{booktabs}
\usepackage{fullpage}
\usepackage{nicefrac}
\usepackage{xspace}
\usepackage[noadjust]{cite}
\usepackage{url}\urlstyle{rm}
\usepackage{graphicx}
% \usepackage{graphics}
\usepackage[space]{grffile} 

\usepackage[usenames,dvipsnames]{xcolor}
\usepackage[colorlinks]{hyperref}
\definecolor{linkblue}{rgb}{0.1,0.1,0.8}
\hypersetup{colorlinks=true,linkcolor=linkblue,filecolor=linkblue,urlcolor=linkblue,citecolor=linkblue}
% \usepackage[algo2e,ruled,vlined]{algorithm2e}


\usepackage{wrapfig}
\usepackage{pdfpages}




% --- Code part
\usepackage{listings}
\lstset{language=Python} % Changer eventuellement le nom du langage
\newcommand{\class}[1]{\texttt{#1}}

% --- Math part
\newtheorem{theorem}{Theorem}
\newtheorem{lemma}[theorem]{Lemma}
\newtheorem{proposition}[theorem]{Proposition}
\newtheorem{corollary}[theorem]{Corollary}
\newtheorem{definition}[theorem]{Definition}
\newtheorem{algorithm}[theorem]{Algorithm}
\newtheorem{remark}[theorem]{Remark}



% Mathematic abbreviations

\newcommand{\N}{\mathbb{N}}
\newcommand{\R}{\mathbb{R}}
\newcommand{\Z}{\mathbb{Z}}

\newcommand{\Esp}{\mathbb{E}}
\newcommand{\Prob}{\mathbb{P}}
\newcommand{\Var}{\text{Var}}

\newcommand{\A}{\frak{A}}
\newcommand{\F}{\mathbb{F}}


\renewcommand{\epsilon}{\varepsilon}


% \newcommand{\GF}[1]{\text{GF}(#1)}
\newcommand{\GF}[1]{\F_{#1}}


% More definitions
\usepackage{showkeys}
\usepackage{mathtools}
%\usepackage{multirow}
%\usepackage{appendix}



\author{Sylvain Colin \& Gaspard Férey}
\title{The Sidelnikov-Shestakov's Attack applied to the Chor-Rivest Cryptosystem}

\begin{document}

\maketitle
\tableofcontents


\begin{abstract}

In this article, we discuss about the Sidelnikov-Shestakov attack on cryptosystems based on Reed-Solomon codes. Then we describe how this algorithm can be used to improve the attack to the Chor-Rivest Cryptosystem proposed by Vaudenay \cite{Vau01}.

\end{abstract}



\newpage
\section{Introduction}
\label{sec:intro}



\subsection{Our Work}




\newpage
\section{Preliminaries}
\label{sec:Prel}


\subsection{A cryptosystem based on Reed-Solomon codes}

We study here the public-key cryptosystem introduced by Sidelnikov and Shestakov \cite{SidelShes92} applied to the generalized Reed-Solomon codes. Let $\F_q$ be a finite field with $q = p^h$ elements and $\F = \F_q \cup \{ \infty \}$, where $\infty$ has usual properties ( $1/\infty = 0$, etc). We call $\A$ the following matrix:
$$ \A(\alpha_1, \ ... \ , \alpha_n, z_1, \ ... \ , z_n) := \left(
\begin{array}{cccc}
z_1 \alpha_1^0 &  z_2 \alpha_2^0 & \cdots & z_n \alpha_n^0 \\
z_1 \alpha_1^1 &  z_2 \alpha_2^1 & \cdots & z_n \alpha_n^1 \\
 & & \ddots & \\
z_1 \alpha_1^{k-1} &  z_2 \alpha_2^{k-1} & \cdots & z_n \alpha_n^{k-1}
\end{array}
\right) \in \mathcal{M}_{\F_q}(k,n) $$
where $\alpha_i \in \F$ and $z_i \in \F_q\backslash\{0\}$ for all $i \in \{1,...,n\}$. Note that, if $\alpha_i = \infty$, we replace the $i^{th}$ column by the vector $z_i(0,...,0,1)^T$, so that all the coefficients of the matrix are finite.

\bigskip

In the considered cryptosystem, the secret key consists of
\begin{itemize}
\item The set $\{\alpha_1,... ,\alpha_n\}$;
\item The set $\{z_1,... ,z_n\}$;
\item A random nonsingular $k\times k$-matrix $H$ over $\F_q$.
\end{itemize}

The public key is
\begin{itemize}
\item The representation of the field $\F_q$, that is the polynomial used to define $\F_q$ over $\F_p$;
\item The two integers $k$ and $n$ such that $ 0 < k < n \leq q$.
\item $M := H \cdot \A(\alpha_1, \ ... \ , \alpha_n, z_1, \ ... \ , z_n)$.
\end{itemize}

The codewords are then the vectors $c=b.M$ where $b\in\F_q^k$. So, the different codewords have necessarily the following form :

$$c=(z_if_c(\alpha_i))_{1\leq i\leq n}$$ where $f_c$ is a polynomial whose degree is at most $k-1$.


Thus, given a message to send, which is actually a vector $b$ of $\F_q^k$, one will have to transmit the vector $b.M + e$ where $e$ is a random vector of $\F_q^n$ with Hamming weight at most $t=\lfloor\frac{n-k}{2}\rfloor$.
So, since a GRS code correct at most $t=\lfloor\frac{n-k}{2}\rfloor$ error, the original message can be recovered by computing $b'=b.M$, finding the closest codeword from the received message, and then computing $b'M^{-1}$.
However, the original message can not be easily recovered when not knowing the GRS code used in the secret key.

\subsection{Equivalence between Reed-Solomon codes}

Sidelnikov and Shestakov show \cite{SidelShes92} that for all $a \in \F_q\backslash\{0\}$ and $b \in \F_q$, there exists $H_1, H_2, H_3 \in \mathcal{M}_{F_q}(k,k)$ invertible such that
\begin{eqnarray*}
H_1 \A(a\cdot\alpha_1 + b, \ ... \ , a\cdot \alpha_n + b, c_1z_1, \ ... \ , c_n z_n) &=& \A(\alpha_1, \ ... \ , \alpha_n, z_1, \ ... \ , z_n) \\
H_2 \A\left( \frac{1}{\alpha_1}, \ ... \ , \frac{1}{\alpha_n}, d_1z_1, \ ... \ , d_n z_n \right) &=& \A(\alpha_1, \ ... \ , \alpha_n, z_1, \ ... \ , z_n) \\
H_3 \A\left( \alpha_1, \ ... \ , \alpha_n, a\cdot z_1, \ ... \ , a\cdot z_n \right) &=& \A(\alpha_1, \ ... \ , \alpha_n, z_1, \ ... \ , z_n)
\end{eqnarray*}
This means that for any cryptosystem $M = H \A(\alpha_1, \ ... \ , \alpha_n, z_1, \ ... \ , z_n)$, for any birationnal transformation
$$ \phi : x \mapsto \frac{ax+b}{cx+d} $$
$M = H_{\phi} \A(\phi(\alpha_1), \ ... \ , \phi(\alpha_n), z'_1, \ ... \ , z'_n)$ and by using the unique transformation $\phi$ that maps $(\alpha_1, \alpha_2, \alpha_3)$ to $(0,1,\infty)$, we get that for any cryptosystem $M = H \A(\alpha_1, \ ... \ , \alpha_n, z_1, \ ... \ , z_n)$, $M$ can be uniquely written
$$ M = H' \A(0, 1, \infty, \alpha'_4, \ ... \ , \alpha'_n, z'_1, \ ... \ , z'_n) $$
with $H'$ nonsingular, $z'_i \neq 0$ and $\alpha_i$ distincts elements of $\F_q-\{0, 1, \infty)$.

So, when $M$ is given, it is impossible to compute the original matrices $\A$ and $H$ since many pairs of such matrices lead to the same public matrix $M$. However, computing an equivalent pair is sufficient
since it will allow to decipher the messages as well as the original secret pair of matrices. So, the attack will consist of finding $H$ and $\A(0, 1, \infty, \alpha'_4, \ ... \ , \alpha'_n, z'_1, z'_2, \ ... \ , z'_n)$, equivalent to the original pair. 
We can also assume that $z'_1=1$. Indeed, if we multiply all the elements $z'_i$ by a factor $a \in \F_q$ and all the elements of the matrix $H$ by $a^{-1}$, the resulting matrix $M$ will be the same.


\newpage
\section{The Sidelnikov-Shestakov Attack}
\label{sec:SSattack}

The attack of Sidelnikov-Shestakov consists of the following steps.
\bigskip

First we assume that the public key is as described in the previous section :
$$ M = H' \A(0, 1, \infty, \alpha'_4, \ ... \ , \alpha'_n, 1, z'_2, \ ... \ , z'_n) $$

We compute then the echelon form of $M$.
$$ E(M) = 
\left(
\begin{array}{ccccccc}
1 & 0 & \cdots & 0 & b_{1,k+1} & \cdots & b_{1,n} \\
0 & 1 & \cdots & 0 & b_{2,k+1} & \cdots & b_{2,n} \\
  &   & \ddots &   & \vdots &   & \vdots \\
0 & \cdots & 0 & 1 & b_{k,k+1} & \cdots & b_{k,n}
\end{array}
\right) = H'' \cdot M
$$

Since the echelon form can be computed only with left multiplication of the matrix M, the $k$ lines of $E(M)$ are codewords. As a consequence, if we call $f_i$ the polynomial associated to the $i^{th}$
line, we have :
\begin{itemize}
\item $\forall$  $1\leq i\leq k, f_i(\alpha_i)=1 $
\item $\forall$  $1\leq i\neq j\leq k, f_i(\alpha_j)=0 $
\item $\forall$  $1\leq i\leq k$ $\forall$ $k+1\leq j\leq n$, $f_i(\alpha_j)=b_{i,j} $
\end{itemize}
So, since all the $\alpha_i$ are different, the polynomial $f_i$ has $k-1$ simple roots. As a consequence, $b_{i,j}\neq0$ for all $1\leq i\leq k$ and $k+1\leq j\leq n$. Moreover, we know the general 
form of the polynomial $f_i$ :
$$f_i(X) = c_i.\prod_{1\leq j\leq k, i\neq j} (X-\alpha_j)$$ where $c_i \in \F\backslash{0}$.

For $2 \leq k \leq n-2$, this attack works with a complexity of ...


\newpage
\section{Application to the Chor-Rivest Cryptosystem}
\label{sec:CRcrypt}

\subsection{The Chor-Rivest Cryptosystem}

Secret keys consist of
\begin{itemize}
\item an element $t \in \GF{q}$ with algebraic degree $h$
\item a generator $g$ of $\GF{q}^{\ast} $
\item an integer $d \in \Z_{q-1} $
\item a permutation $\pi$ of $\{ 0, ... , p-1 \}$.
\end{itemize}

Public keys consist of all
$$ c_i = d + \log_g(t + \alpha_{\pi(i)}) \mod q-1 $$

The message consists in a bitstring $m = [m_0...m_{p-1}]$ of length $p$ such that $\sum_i m_i = h$. The ciphertext is
$$ E(M) := \sum_{i=0}^{p-1} m_i c_i$$
To decipher this message, we compute
$$ g^{E(M) - hd} =  \prod_i \left( t + \alpha_{pi(i)}\right)^{c_i} $$

% Explain here how to retrieve then the c_i



When we attack this cryptosystem, we can consider a generator $g_0 = g^u$ with $u$ unknown and $\gcd(u, q-1) = 1$ we then have
$$ g_0^{c_i} = \left( g^d \left( t + \alpha_{\pi(i)} \right) \right)^u = \left(A + \alpha_{\pi(i)} \cdot B\right)^u$$
We can then consider that the secret key is
\begin{itemize}
\item $A \in \F_q$.
\item $B \in \F_q$ such that $t = A\cdot B^{-1}$ has algebraic degree $h$.
\item $0 < u < q-1$ prime with $q-1$.
\item the permutation $\pi$ of $\{ 0, ... , p-1 \}$.
\end{itemize}
and public key consists in all the
$$ d_i := \left(A + \alpha_{\pi(i)} \cdot B\right)^u \in \F_q$$
The ciphertext becomes
$$ E'(M) := \prod_{i=0}^{p-1} d_i^{m_i} = g^{uE(M)} = B^{uh} \left( \prod_i \left( t + \alpha_{pi(i)}\right)^{c_i} \right)^u$$
Knowing $u$, $B$ and $h$, it is easy to compute from $E'(M)$, the following quantity
$$ \prod_i \left( t + \alpha_{pi(i)}\right)^{c_i} $$
which allow us to retrieve all the $c_i$.


\subsection{Link with Reed-Solomon codes}

Trying to attack this cryptosystem show some relations between this problem and the previous one studied in section \ref{sec:Prel}. In particular we have the following theorem.

\begin{theorem}
\label{thm:link}
Let $2 \leq k \leq p-2$.
Suppose there exists $(Q_i)_{0 \leq i \leq k-1}$ $k$ polynomials of $\GF{p}[X]$ linearly independent with degree smaller than $k-1$.
Suppose the evaluations $m_{i,j} := Q_i(\alpha_{\pi(j)})$ is known for all $i$ and $j$.
Then the permutation $\pi$ can be recovered in polynomial time using a Sidelnikov-Shestakov attack on the matrix $M = (m_{i,j})_{i,j} \in \mathcal{M}_{k,p}(\GF{p})$.
\end{theorem}
\begin{proof}
We suppose here that one of the $Q_i$ has a degree exactly $k$.
Then we write the square non singular matrix $H = (h_{i,j}) \in \mathcal{M}_k(\GF{p})$ of the coefficients of the $Q_i$
$$ Q_i(X) = \sum_{j=0}^{k-1} h_{i,j} X^j $$
If we still consider
$$ \A_k := \left( \alpha_{\pi(j)}^i \right)_{0 \leq i < k, 0 \leq j \leq p-1} \in \mathcal{M}_{k,p}(\GF{p})$$
We have the equality
$$ H \cdot \A_k = M$$
with $H$ non singular and since $k \leq p-2$, this is exactly the public key of a cryptosystem based on the Reed-Solomon codes described in section \ref{sec:Prel}.
\end{proof}

So a possible way to attack the Chor-Rivest cryptosystem would be to find the evaluations of enough small degree polynomials in the $\alpha_{\pi(i)}$.



\subsection{A First Attack using Reed-Solomon codes}

We have for all $j$
$$ g^{c_j} = g^d \cdot (t + \alpha_{\pi(j)} ) = A + \alpha_{\pi(j)} \cdot B $$
where $\alpha_{\pi(j)} \in \GF{p}$ and $A$ and $B$ are elements of $\GF{p^h}$.

A naive attack would be then to try to guess at random the generator $g$. We will see that although finding the precise $g$ is very unlikely, there is a family of generators that can still allow us to retrieve $\pi$.

\subsection{Small power of $g$}

As an attempt to guess $g$, we can choose a random generator $g_0$ of $\GF{q}^{*}$. We have $g_0 = g^u$ and
$$ g_0 ^{c_j} = (A + \alpha_{\pi(j)} \cdot B)^u $$
if we write this quantity in a certain base $(e_i)_{1 \leq i \leq h}$ of $\GF{p^h}$, we notice that each coordinate is a polynomial $Q_i$ in the $\alpha_{\pi(j)}$.
$$ g_0^{c_j} = \sum_{i=1}^h Q_i(\alpha_{\pi(j)}) e_i $$
where $Q_i$ has its coefficients in $\GF{p}$.
$Q_i$ depends on $A$, $B$, $u$ and obviously on $i$. However, $Q_i$ does not depend on $j$.

Besides, we have
\begin{itemize}
\item $\deg Q_i \leq u$
\item $\deg Q_i < p$ since $\alpha_{\pi(j)}^p = \alpha_{\pi(j)}$.
\end{itemize}

This means that we have access to the evaluations in the $\alpha_{\pi(j)}$ of $h$ polynomials of degree smaller than $u$. According to Theorem~\ref{thm:link}, a sufficient condition for this attack to work is $u \leq h-1 \leq p-3$. The last inequality is most likely true since $h$ is chosen close to $p / \log p$. However there are only $h-1$ different elements of $\GF{p^h}$ fulfilling the first one.
This only slightly improves the exhaustive research of $g$.


\subsection{Wider set of generators}

We can notice that if $u = u'p$,
$$ g_0^{c_j} = (A + \alpha_{\pi(j)} \cdot B)^{u'p} = (A^p + \alpha_{\pi(j)} \cdot B^p)^{u'} $$
and the coordinates of this quantity are polynomials of degree $u'$ in $\alpha_{\pi(j)}$.
This also means that if $u$ is written $u = \sum_{i=0}^{h-1} u_i p^i$ in base $p$, then
$$ g_0^{c_j} = \prod_{i=0}^{h-1} (A^{p^i} + \alpha_{\pi(j)} \cdot B^{p^i} )^{u_i} $$
whose coordinates are a polynomial of degree $w_p(u) := \sum_{i=0}^{h-1} u_i$ in the $\alpha_{\pi(j)}$.

This mean that all $u$ such that $w_p(u) < h$ allow to retrieve the permutation and break the cryptosystem.
The number of such $u$ is
$$ \left( \binom{h+1}{h-1} \right) = \binom{2h}{h-1} =\Theta\left( 4^h \sqrt{h} \right) $$
This is a drastic improvement in the exhaustive research of $g$. However this remains quite small compared to the number $\phi(p^h-1)$ of different generators in $\GF{p^h}$ which is comparable to $p^h$.




\newpage
\section{Vaudenay attack}
\label{sec:Vau}

We can see that the previous attack requires to find a generator of $\F_q$ among the elements that can be written $g^u$ with $w_p(u) \leq h$. Since there are very few of such elements compared to the $\phi(p^h)$ different generators of $\F_q$, Vaudenay suggests \cite{Vau01} to consider a generator $g_{p_r}$ of the sub-field $\F_{p^r}$ of $\F_{p^h}$. He introduces the following theorem
\begin{theorem}
For any factor $r$ of $h$, there exists a generator $g_{p^r}$ of the multiplicative group of the subfield $\F_{p^r}$ of $\F_q$ and a polynomial $Q$ with degree $h/r$ whose coefficients are in $\F_{p^r}$ and such that $-t$ is a root and for all $j$ , we have $Q(\alpha_{\pi(j)}) = g_{p^r}^{c_j}$.
\end{theorem}

If we chose a base $(e_i)_{1 \leq i \leq r}$ of $\GF{p^r}$, we can write the coefficients of $g_{p^r}^{c_j}$ in this base as polynomials $Q_i$ in $\alpha_{\pi(j)}$. We get
$$ g_{p^r}^{c_j} = \sum_{i=1}^r Q_i(\alpha_{\pi(j)}) e_i $$
with $\deg Q_i \leq h/r$. We get the evaluation of $r$ polynomials of degree smaller than $h/r$.

This means that instead of searching a generator among the approximately $p^h$ generators of $\F_q$, we could search only within $\F_{p^r}$ with $r$ as small as possible. We notice that to be able to apply Theorem~\ref{thm:link}, we must have $h/r < r$. This yields the following theorem.

\begin{theorem}
When $r > \sqrt{h}$, there exists a polynomial "known $g_{p^r}$" attack on the Chor-Rivest cryptosystem.
\end{theorem}

This theorem is basically the main result from Vaudenay's article \cite{Vau01}. It states that to attack the Chor-Rivest cryptosystem, one can only search for the generator $g_{p^r}$ among $\Theta\left(p^r\right)$ possible choices instead of the exhaustive research for $g$ ($\Theta\left(p^h\right)$ choices).

However the "known $g_{p^r}$ attack" suggested in Vaudenay's article can be improved in two ways.

\begin{itemize}
\item First it is only polynomial when $r > \sqrt{h}$. We will see that using a Reed-Solomon attack, we can still retrieve $\pi$ in polynomial time provided we manage to get enough linearly independent polynomials in the $\alpha_{\pi(j)}$ which is possible even for small values of $r$ using $g_{p^r}^u$ with $u$ small enough.
\item Besides, the Reed-Solomon attack doesn't require the knowledge of all the $c_j$. Only $O(h)$ (or a little more when considering the attack for $r$ small) of them. So actually only a small proportion of them is enough. This makes this attack strong.
\end{itemize}




\subsection{Generating more rows...}
When we find $g_{p^r}$ such that $g_{p^r}^{c_j} = Q(\alpha_{\pi(j)})$ where $\deg Q \leq h/r$, we only have $r$ rows corresponding to the $r$ different polynomials of the coordinates of $g_{p^r}^{c_j}$ in a certain base of $\F_{p^r}$. Being able to generate more row would allow to chose a lower $r$ and improve drastically the attack.

We could consider now the coordinates of $g_{p^r}^{u c_j}$ for $u$ such that $w_p(u)$ remains small. For example, we could decide $w_p(u) \leq w$.

This should yield (assuming $w \gg r$) up to
$$ \left( \binom{r+1}{w} \right) = \binom{r+w+1}{w} = \binom{r+w+1}{r+1} = \Theta\left(\frac{w^{r+1}}{(r+1)!}\right) $$
different polynomials of degree at most $w \cdot \frac{h}{r}$.
Unfortunately, it seems probable that these are strongly linearly dependent...

For example, the coordinates of $g_{p^r}^{p c_j}$ are linearly dependent on the coordinates of $g_{p^r}^{c_j}$. Indeed, when decomposed in a normal base of $\F_{p^r}$, these two sets of vectors of coordinates only differs by a rotation.


\subsection{Simulation}

We run a simulation with the following parameters
\begin{itemize}
\item $p = 197$, $h = 24$, $r = 3 < \sqrt{h}$.
\item We define $\F_q$ as the quotient of $\F_p[X]$ by the polynomial
\begin{eqnarray*}
X^{24} &+& 192X^{23} + 152X^{22} + 25X^{21} + 75X^{20} + 67X^{19} + 92X^{18} + 23X^{17} + 45X^{16} + 97X^{15} \\
 &+& 2X^{14} + 21X^{13} + 106X^{12} + 130X^{11} + 128X^{10} + 136X^9 + 195X^8 + 95X^7 + 155X^6 \\
 &+& 34X^5 + 51X^4 + 180X^3 + 97X^2 + 23X + 87
\end{eqnarray*}
\item We choose $g := X + 2$ the private multiplicative generator.
\item We compute
\begin{eqnarray*}
g_{p^r} &=& g^{\frac{p^h-1}{p^r-1}} \\
&=& 153X^{23} + 168X^{22} + 167X^{21} + 45X^{20} + 128X^{19} + 68X^{18} + 103X^{17} + 11X^{16}\\
&&+ 139X^{15} + 190X^{14} + 75X^{13} + 73X^{12} + 190X^{11} + 64X^{10} + 173X^9 + 34X^8 \\
&&+ 88X^7 + 30X^6 + 139X^5 + 146X^4 + 111X^3 + 80X^2 + 136X + 48\\
\end{eqnarray*}
\item We choose different values for $d$, $t$ and $\pi$, the results remain the same.
\item We choose the base $\left(g_{p^r}, g_{p^r}^p, g_{p^r}^{p^2} \right)$ for the $\F_p$-vector space $\F_{p^r}$ (but this is of no influence on the results).
\item We choose $(u_i)_{1\leq  i \leq 11} = (1,2,p+1,3,2p+1,p+2,4,p+3,3p+1,2p+2,2p^2+p+1)$\\
We have $w_p(u_i) \leq 4$ so the coordinates of $g_{p^r}^{u_i c_j}$ are polynomials of degree smaller than $4 h/r+1 = 33$ in the $\alpha_{\pi(j)}$.
\end{itemize}
As a result, we obtain $11 \times 3 = 33$ lines of coordinates linearly independent (the simulation is the verification of this independence). This would allow the attack to retrieve the permutation $\pi$ using the attack on the cryptosystem based on Reed-Solomon codes.

This proves that it is possible to duplicate the number of lines at the expense of the degree of the polynomials considered. Therefore the condition $r \geq \sqrt{h}$ is not absolutely necessary and we can hope to get a (far) smaller lower bound on $r$.

On the following tabular, we present the results of several simulations for different value of $p$, $h$. Each time, we choose the smallest value for $r$ that allow an attack on the system. We display here the maximum weight $w$ chosen for the exponents $u_i$ of $g_{p^r}$ and the number of linearly independent lines these $g_{p^r}^{u_i}$ allow to generate (should be $w\cdot \frac{h}{r} +1$).


\begin{tabular}{|c|c|c|c|l|}
\hline
$p$ & $h$ & $r$ & $w_p(u_i)$ & Number of linearly independent lines  \\
\hline
197 & 24 & 2 & $w$ & $\frac{w(w+3)}{2} < 12w+1$ Attack impossible. \\
\hline
197 & 24 & 3 & 4 & 33\\
\hline
197 & 24 & 4 & 2 & 13\\
\hline
197 & 24 & $r \geq 6$ & 1 & $1 + h/r$ \\
\hline
193 & 36 & 2 & $w$ & $\frac{w(w+3)}{2} < 18w+1$ Attack impossible. \\
\hline
193 & 36 & 3 & 6 & 73\\
\hline
193 & 36 & 4 & 3 & 28\\
\hline
193 & 36 & 6 & 2 & 13\\
\hline
193 & 36 & $r \geq 9$ & 1 & $1 + h/r$ \\
\hline
251 & 60 & 2 & $w$ & $\frac{w(w+3)}{2} < 30w+1$ Attack impossible. \\
\hline
251 & 60 & 3 & 8 & 161\\
\hline
251 & 60 & 4 & 4 & 61\\
\hline
251 & 60 & 5 & 3 & 37\\
\hline
251 & 60 & 6 & 2 & 21\\
\hline
251 & 60 & $r \geq 10$ & 1 & $1 + h/r$ \\
\hline
\end{tabular}

When we choose, $r = 2$ and $w_p(u_i) \leq w$, the number of linearly independent lines is $\frac{w(w+3)}{2}$. This does not allow to attack the cryptosystem since to have $\frac{w(w+3)}{2} \geq 12w+1$, we would need $w \geq 22$ and if $w_p(u_i) = 2$, then the degree of the corresponding polynomials is $22 \times 12+1 = 265 > 197$.

Regarding that last set of examples, the Chor-Rivest attack would require to do an exhaustive research on approximately $251^{10}$ different possible generators of $\F_{p^{10}}$. With our approach, the cryptosystem can be attacked using the right generator $g_{p^3}$ of $\F_{p^3}$. The research is clearly a lot faster.

\subsubsection{Number of linearly independent polynomials}

We can study now the number of linearly independent polynomials that the writing of $g_{p^r}^{uc_j}$ can generate for all $u$ such that $w_p(u) \leq w$.

For example, for $r = 2$, it seems (experimentally) that the number of such linearly independent polynomials is always $\frac{w(w+3)}{2}$ and does not depend on $p$ or $h$.

For $r = 3$, the number of such linearly independent polynomials is $\frac{(w+1)(w+2)(w+3)}{6}-1$.

We could guess that for bigger $r$, this number is always $\Theta\left(\frac{w^r}{r!}\right)$.


\section{Draft}

We should prove that indeed at least $C\frac{w^r}{r!}$ linearly independent polynomials can be found in the coordinate of $g_{p^r}^{uc_j}$ with $w_p(u) \leq w$.

Then, we have that all we need for the attack to work is to choose $r$ such that there exists $w$ such that $w \cdot \frac{h}{r} + 1 \leq p-2$ and


\begin{eqnarray*}
C\frac{w^r}{r!} &\geq& w \frac{h}{r} \\
\Leftrightarrow C w^{r-1} &\geq& h\cdot (r-1)! \\
\Leftarrow C w^{r-1} &\geq& h (r-1)^{r-1} \\
\Leftarrow w &\geq& (r-1) \left(\frac{h}{C}\right)^{\frac{1}{r-1}} \geq C' r h^{\frac{1}{r-1}}
\end{eqnarray*}
This yield
\begin{eqnarray*}
C' h \cdot h^{\frac{1}{r-1}} &\leq& p-3 \\
\frac{\log h}{r-1} &\leq& \log (p-3) - \log h + C'' \\
r-1 &\geq& \frac{\log h}{\log(p-3) - \log h + C''} 
\end{eqnarray*}
besides, $h$ must be chosen such that
$$ h \simeq \frac{p}{\log p}$$
so
$$ r \geq \frac{\log p - \log \log p}{C'' + \log \log p} \geq C''' \frac{\log p}{\log\log p}$$
meaning that the research for a generator $g_{p^r}$ is almost polynomial...




\section{Conclusions}

We can notice that this attack on Chor-Rivest cryptosystem is only effective when $h$ possesses a small factor. For example, L. Hernandez Encinas \textit{et alii} \cite{Enc04} suggest to use $h$ prime. Such a cryptosystem remains however useless nowadays essentially because of the complexity of the discrete logarithm problem.





\bibliographystyle{abbrv}
\bibliography{references}
\newpage
\appendix
\end{document}

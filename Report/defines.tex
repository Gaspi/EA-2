
\usepackage{amsmath, amsxtra, amsfonts, amssymb, amstext, amsthm}

\usepackage{booktabs}
%\usepackage{fullpage}
%\usepackage{nicefrac}
\usepackage{xspace}
\usepackage[noadjust]{cite}
\usepackage{url}\urlstyle{rm}
\usepackage{graphicx}
% \usepackage{graphics}
\usepackage[space]{grffile} 

\usepackage[usenames,dvipsnames]{xcolor}
\usepackage[colorlinks]{hyperref}
\definecolor{linkblue}{rgb}{0.1,0.1,0.8}
\hypersetup{colorlinks=true,linkcolor=linkblue,filecolor=linkblue,urlcolor=linkblue,citecolor=linkblue}

% \usepackage[algo2e,ruled,vlined]{algorithm2e}
% \usepackage{algpseudocode}


%\usepackage{wrapfig}
\usepackage{pdfpages}
\usepackage{tabularx}




% --- Code part
\usepackage{listings}
\lstset{language=Python} % Changer eventuellement le nom du langage
\newcommand{\class}[1]{\texttt{#1}}

% --- Math part
\newtheorem{theorem}{Theorem}
\newtheorem{lemma}[theorem]{Lemma}
\newtheorem{proposition}[theorem]{Proposition}
\newtheorem{corollary}[theorem]{Corollary}
\newtheorem{definition}[theorem]{Definition}
\newtheorem{algorithm}[theorem]{Algorithm}
\newtheorem{remark}[theorem]{Remark}
\newtheorem{postulate}[theorem]{Postulate}


% Mathematic abbreviations

\newcommand{\N}{\mathbb{N}}
\newcommand{\R}{\mathbb{R}}
\newcommand{\Z}{\mathbb{Z}}

\newcommand{\Esp}{\mathbb{E}}
\newcommand{\Prob}{\mathbb{P}}
\newcommand{\Var}{\text{Var}}

\newcommand{\A}{G}
\newcommand{\F}{\mathbb{F}}

% \newcommand{\GF}{\text{GF}}


\renewcommand{\epsilon}{\varepsilon}

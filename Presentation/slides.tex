% !TEX encoding = UTF-8 Unicode
\documentclass[]{beamer}
\usetheme{Boadilla}
%\usetheme{Copenhagen}

\usepackage[utf8]{inputenc}
\usepackage[T1]{fontenc}


\usepackage{amsmath, amsxtra, amsfonts, amssymb, amstext, amsthm}

\usepackage{booktabs}
%\usepackage{fullpage}
%\usepackage{nicefrac}
\usepackage{xspace}
\usepackage[noadjust]{cite}
\usepackage{url}\urlstyle{rm}
\usepackage{graphicx}
% \usepackage{graphics}
\usepackage[space]{grffile} 

\usepackage[usenames,dvipsnames]{xcolor}
\usepackage[colorlinks]{hyperref}
\definecolor{linkblue}{rgb}{0.1,0.1,0.8}
\hypersetup{colorlinks=true,linkcolor=linkblue,filecolor=linkblue,urlcolor=linkblue,citecolor=linkblue}
% \usepackage[algo2e,ruled,vlined]{algorithm2e}


%\usepackage{wrapfig}
\usepackage{pdfpages}
\usepackage{tabularx}




% --- Code part
\usepackage{listings}
\lstset{language=Python} % Changer eventuellement le nom du langage
\newcommand{\class}[1]{\texttt{#1}}

% --- Math part
\newtheorem{theorem}{Theorem}
\newtheorem{lemma}[theorem]{Lemma}
\newtheorem{proposition}[theorem]{Proposition}
\newtheorem{corollary}[theorem]{Corollary}
\newtheorem{definition}[theorem]{Definition}
\newtheorem{algorithm}[theorem]{Algorithm}
\newtheorem{remark}[theorem]{Remark}
\newtheorem{postulate}[theorem]{Postulate}


% Mathematic abbreviations

\newcommand{\N}{\mathbb{N}}
\newcommand{\R}{\mathbb{R}}
\newcommand{\Z}{\mathbb{Z}}

\newcommand{\Esp}{\mathbb{E}}
\newcommand{\Prob}{\mathbb{P}}
\newcommand{\Var}{\text{Var}}

\newcommand{\A}{G}
\newcommand{\F}{\mathbb{F}}

% \newcommand{\GF}{\text{GF}}


\renewcommand{\epsilon}{\varepsilon}



\title[Attaque de Chor-Rivest]{Attaque de Sidelnikov-Shestakov appliquée au cryptosystème de Chor-Rivest }

\subtitle{INF 581 \ - \ Enseignement d'Approfondissement \\ D. Augot}

\author[S. Colin \& G. Férey]{Sylvain Colin \& Gaspard Férey}

\institute[X 2011]{Département d'Informatique\\ Ecole Polytechnique, France }
\date{27 Mars 2014}


%\logo{\includegraphics[height=1cm]{pictures/logo.png}}
\titlegraphic{\includegraphics[height=1.5cm]{pictures/logo.png}}




\begin{document}


%------- the titlepage frame ---------------%
\begin{frame}[plain]
  \titlepage
\end{frame}


\section{Attaque de Sidelnikov-Shestakov}

\subsection{Cryptosystème de McEliece utilisant les codes de Reed-Solomon}

\begin{frame}{Cryptosystème de McEliece utilisant les codes de Reed-Solomon}

Clef privée
\begin{itemize}
\item Une matrice $G = G(\alpha_1, \ ... \ , \alpha_n , z_1 , \ ... \ z_n) = (z_i\alpha_i^{j})_{i=1...n, j=0...k-1}$
\item Une matrice inversible $H$ de taille $k\times k$ dans $\F_q$.
\end{itemize}
Clef publique
\begin{itemize}
 \item La matrice $M=H\cdot G$.
 \item L'entier $t=\lfloor\frac{n-k}{2}\rfloor$.
\end{itemize}
Messages originaux : vecteurs de $b\in \F_q^{k}$.

Message chiffré : $b\cdot M + e$ avec $e$ de poids de Hamming inférieur à t et $b\cdot M=(z_if_b(\alpha_i))_{1\leq i\leq n}$ ($f_b$ de degré au plus $k-1$).

Déchiffrement :
\begin{itemize}
 \item On calcule $b\cdot H$ par un algorithme de déchiffrement de code GRS.
 \item On calcule $b$ par multiplication par $H^{-1}$
\end{itemize}

\end{frame}


\subsection{Attaque de Sidelnikov et Shestakov}

\begin{frame}{Attaque de Sidelnikov et Shestakov}
\begin{itemize}
 \item Basée sur l'équivalence entre codes GRS : $\exists H',(\alpha'_3, \ ... \ \alpha'_{k}, \alpha'_{k+2},\ ... \ , \alpha'_n)$ et $(z'_1, \ ... \ z'_{k},z'_{k+2}, \ ... \ z'_n)$ tels que
 $$H\cdot G=H'\cdot G(0,1,\alpha'_3, \ ... \ ,\infty ,\alpha'_{k+2},\ ... \ , \alpha'_n,z'_1, \ ... \ ,1,z'_{k+2}, \ ... \ z'_n)$$
 \item On calcule la forme échelon de M :

$$ E(M) = 
\left( \ \ I_k \ \ \left| \ \ \left( b_{i,j}\right)_{1\leq i\leq k, k+1 \leq j \leq n} \ \
\right. \right)
$$
 \item On remarque que :
 $$f_{b_i}(X) = c_{b_i}\cdot \prod_{1\leq j\leq k, i\neq j} (X-\alpha_j)$$ avec $c_{b_i}=b_{i,k+1}$ et $b_{i,j}=z_if_{b_j}(\alpha_i)$.
\end{itemize}

\end{frame}

\begin{frame}{Attaque de Sidelnikov et Shestakov}
Calcul des $\alpha_i$ :
\begin{itemize}
 \item $\forall k+2\leq j\leq n, \alpha_j = \frac{b_{2,j}\cdot c_{b_1}}{b_{2,j}\cdot c_{b_1} - b_{1,j}\cdot c_{b_2}}$
 \item $\forall 3\leq i\leq k, \alpha_i = \alpha_{k+2} - \frac{b_{i,k+2}}{b_{1,k+2}}\cdot \frac{c_{b_1}}{c_{b_2}}\cdot (\alpha_{k+2}-1)$
 \item On calcule un ensemble de $\alpha'_i$ équivalent et tous finis en trouvant un élément $\alpha$ différent de tous les $\alpha_i$ et en appliquant la transformation birationnelle $ \phi : x \mapsto \frac{1}{x - \alpha}$
\end{itemize}
Calcul des $z_i$ :
\begin{itemize}
 \item On note $L_i(X) = \prod_{1\leq j\leq k, i\neq j} (X-\alpha_j) = \frac{1}{c_{b_i}}\cdot f_{b_i}(X)$
 \item $\forall 1\leq i\leq k, z_i = \frac{L_i(\alpha_{k+1})}{b_{i,k+1}\cdot L_i(\alpha_i)}$
 \item $\forall k+2\leq j\leq n, z_j = \frac{b_{1,j}}{b_{1,k+1}} \cdot \frac{L_1(\alpha_{k+1})}{L_1(\alpha_j)}$
\end{itemize}
Calcul de $H$ : $H = M_{k}\cdot G_{k}^{-1}$
\end{frame}


\section{Cryptosystème de Chor-Rivest}


\subsection{Le cryptosystème de Chor-Rivest}
\begin{frame}{Le cryptosystème de Chor-Rivest}
Clef privée:
\begin{itemize}
\item $t \in \F_q$ dont le polynôme minimal est de degré $h$.
\item $g$ générateur $\F_q^*$.
\item $0 \leq d < q$.
\item $\pi$ permutation de $\{ 0, ... , p-1 \}$.
\end{itemize}
Clef publique:
$$ c_i := d + \log_g(t + \alpha_{\pi(i)}) \mod q-1 $$
Message $m = [m_0...m_{p-1}]$ avec $\sum_i m_i = h$.
Message chiffré:
$$ E(M) := \sum_{i=0}^{p-1} m_i c_i \mod q-1 $$
On déchiffre en calculant
$$ g^{E(M) - hd} =  \prod_i \left( t + \alpha_{\pi(i)}\right)^{m_i} $$
\end{frame}




\subsection{Notre attaque}

\begin{frame}{Lien avec Reed-Solomon}
\begin{theorem}
\label{thm:link}
Pour $2 \leq k \leq p-2$, supposons qu'il existe $k$ polynômes $(Q_i)_{1 \leq i \leq k}$ de $\F_p[X]$ de degré inférieur à $k-1$ linéairement indépendants. Supposons connues les évaluations de ces polynômes en les $\alpha_{\pi(j)}$, $m_{i,j} := Q_i(\alpha_{\pi(j)})$.
Alors la permutation $\pi$ peut être retrouvée en temps polynomial en utilisant une attaque de Sidelnikov-Shestakov sur la matrice $ M = (m_{i,j})_{i,j} \in \mathcal{M}_{k,p}(\F_p)$.
\end{theorem}
\end{frame}


\begin{frame}{Attaque de Vaudenay}
\begin{theorem}
Quelque soit  $r$ divisant $h$, il existe un générateur $g_{p^r}$ du groupe multiplicatif $\F_{p^r}^*$ (où $F_{p^r}$ sous-corps de $\F_q$) et $Q \in \F_{p^r}[X]$ de degré $h/r$ admettant $-t$ pour racine et tel que pour tout $j$, $Q(\alpha_{\pi(j)}) = g_{p^r}^{c_j}$.
\end{theorem}
\begin{proof}
On a $g_{p^r} = g^{\frac{q-1}{p^r-1}}$ et
$$ Q(X) = g_{p^r}^d \prod_{i=0}^{h/r-1} \left( X + t^{p^{ri}} \right) $$
\end{proof}
\end{frame}


\begin{frame}{Attaque de Vaudenay}
\begin{theorem}
Si $r > \sqrt{h}$, et $g_{p^r}$ connu, il existe une attaque du cryptosystème de Chor-Rivest en temps polynomial.
\end{theorem}
\begin{proof}
Les $r$ coordonnées de $g_{p^r}^{c_j}$ sont des polynômes de degré $h/r > r$ en les $\alpha_{\pi(j)}$.
On utilise une attaque de Sidelnikov-Shestakov sur la matrice de ces coordonnées.
\end{proof}
\end{frame}


\begin{frame}{Utilisation des puissances de $g_{p^r}$}
Soit $r$ diviseur de $h$ et $(e_i)_{1 \leq i \leq r}$ une base de $\F_{p^r}$. On note
\begin{itemize}
\item $ U_w := \{ u \in [0,p^r-1] | w_p(u) \leq w \} $
\item $h[i]$ la $i$ème coordonnée de $h \in \F_{p^r}$ dans la base $(e_i)$.
\item On définit $M^{(w)} \in \mathcal{M}_{r \cdot |U_w|, p}$
$$ M^{(w)} := \left(g_{p^r}^{uc_j}[i] \right)_{(i,u) \in [1,r] \times U_w , 1 \leq j \leq p}$$
\item $u_w := \text{rank} \left( M^{(w)} \right)$
\end{itemize}
On a
$$ u_w \leq r \cdot |U_w| = O\left( \frac{w^{r+1}}{r!} \right) $$

\begin{theorem}
Si $u_w = wh/r+1 \leq p-2$, Sidelnikov Shestakov fournit une attaque en temps polynomial.
\end{theorem}

\end{frame}


\subsection{Simulations et condition sur $r$}
\begin{frame}{Postulat}

\begin{postulate}
Pour tout $r > 2$,
$$ u_w = \min \left( \binom{w+r}{r}, w\frac{h}{r} + 1 , p \right).$$
\end{postulate}
Vérifié sur
\begin{center}
\begin{tabular}{|c|c|c|}
\hline
$r$ & $w$ 	& $h/r$ \\
\hline
2	& [1,17]	& \{1,2\}	\\
\hline
3	& [1,17]	& [1,30] \\
\hline
4	& [1,17]	& [1,30] \\
\hline
5	& [1,17]	& [1,30] \\
\hline
\end{tabular}
\end{center}
\end{frame}


\begin{frame}{Condition sur $r$}
On suppose
\begin{itemize}
\item $u_w = \min \left( \binom{w+r}{r}, w\frac{h}{r} + 1 , p \right)$  \ \ \ \ \ \ (pour tout $w$)
\item $h \sim p / \log p$
\item $h$ a de petits diviseurs
\item Il existe $w$ tel que
\begin{eqnarray*}
wh/r+1 &\leq & p-2 \\
wh/r+1 &\leq & u_w 
\end{eqnarray*}
\end{itemize} 
On obtient
$$ r \sim \frac{\log p}{\log \log p} $$
\end{frame}


\subsection{Algorithme et complexité espérée}

\begin{frame}{Algorithme}
\textbf{Input : } Description de $\F_{p^h}$ et la clef publique: $(c_j)_{1 \leq j \leq p}$
\begin{itemize}
\item Calculer le plus petit diviseur $r$ de $h$ qui permette une attaque.

\item Calculer le plus grand $w$ possible et l'ensemble $U_w$.
\item Choisir une base $(e_i)_{1 \leq i \leq r}$ de $\F_{p^r}$ et générer la matrice projetant les éléments de $\F_q$ dans cette base.

\item \textbf{Pour} tout générateur $g_{p^r}$ possible de $\F_{p^r}$ \textbf{faire}
\begin{itemize}
\item Générer $M$ à partir de $wh/r+1$ lignes indépendantes à partir des lignes de $M^{(w)}$

\item \textbf{Si } on peut trouver une ligne de $M^{(w)}$ indépendantes de celles de $M$\\
\textbf{Alors } Passer au générateur suivant. \\
\textbf{Sinon } Sortir de la boucle, retenir $M$ et $g_{p^r}$.
\end{itemize}
\item Effectuer une attaque de Sidelnikov Shestakov attack sur $M$ pour générer toute les permutations possibles $(\pi_i)$.
\item \textbf{Pour chaque } permutation $\pi$ \textbf{ faire }
\begin{itemize}
\item Déchiffrer en utilisant une attaque connaissant $g_{p^r}$ et $\pi$.
\end{itemize}
\end{itemize}
\end{frame}


\begin{frame}{Complexité en temps}
\begin{itemize}
\item Calculs préparatoires: $O\left(p^3\right)$
\item Boucle principale:
$$ (\text{Recherche exhaustive}) \times (\text{"Early abort"}) = O\left( p^{\frac{\log p}{\log \log p} + C} \right) $$
\item Sidelnikov-Shestakov: $O\left(p^3 (\log p)^{O(1)} \right)$
\item Fin de l'attaque: $O\left(p^{O(1)}\right)$
\end{itemize}

\end{frame}


\section{Conclusion}

\begin{frame}{Conclusion}
Attaque de Sidelnikov-Shestakov
\begin{itemize}
\item Réécrite avec les simplifications de Wieshebrink
\item Utilisation de l'infini pour éliminer une recherche de coefficient
\end{itemize}

Notre attaque de Chor-Rivest
\begin{itemize}
\item a une bien meilleure complexité que celle de Vaudenay
\item marche dès que $\Omega(p)$ coefficients $\alpha_{\pi(j)}$ sont connus
\begin{itemize}
\item mais $O(p)$ permutations possibles générées
\end{itemize}
\end{itemize}
\end{frame}

\begin{frame}
\begin{center}
\huge{\textbf{Démonstration de Sidelnikov-Shestakov en sage}}
\end{center}
\begin{figure}
   \centering
   \includegraphics[width=8cm]{pictures/simul.jpg}
\end{figure}
\end{frame}

\begin{frame}
\begin{center}
\huge{\textbf{Questions ?}}
\end{center}
\begin{figure}
   \centering
   \includegraphics[width=8cm]{pictures/questions.png}
\end{figure}
\end{frame}


\end{document}

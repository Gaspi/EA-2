% !TEX encoding = UTF-8 Unicode
\documentclass[]{beamer}
\usetheme{Boadilla}
%\usetheme{Copenhagen}

\usepackage[utf8]{inputenc}
\usepackage[T1]{fontenc}


\usepackage{amsmath, amsxtra, amsfonts, amssymb, amstext, amsthm}

\usepackage{booktabs}
%\usepackage{fullpage}
%\usepackage{nicefrac}
\usepackage{xspace}
\usepackage[noadjust]{cite}
\usepackage{url}\urlstyle{rm}
\usepackage{graphicx}
% \usepackage{graphics}
\usepackage[space]{grffile} 

\usepackage[usenames,dvipsnames]{xcolor}
\usepackage[colorlinks]{hyperref}
\definecolor{linkblue}{rgb}{0.1,0.1,0.8}
\hypersetup{colorlinks=true,linkcolor=linkblue,filecolor=linkblue,urlcolor=linkblue,citecolor=linkblue}
% \usepackage[algo2e,ruled,vlined]{algorithm2e}


%\usepackage{wrapfig}
\usepackage{pdfpages}
\usepackage{tabularx}




% --- Code part
\usepackage{listings}
\lstset{language=Python} % Changer eventuellement le nom du langage
\newcommand{\class}[1]{\texttt{#1}}

% --- Math part
\newtheorem{theorem}{Theorem}
\newtheorem{lemma}[theorem]{Lemma}
\newtheorem{proposition}[theorem]{Proposition}
\newtheorem{corollary}[theorem]{Corollary}
\newtheorem{definition}[theorem]{Definition}
\newtheorem{algorithm}[theorem]{Algorithm}
\newtheorem{remark}[theorem]{Remark}
\newtheorem{postulate}[theorem]{Postulate}


% Mathematic abbreviations

\newcommand{\N}{\mathbb{N}}
\newcommand{\R}{\mathbb{R}}
\newcommand{\Z}{\mathbb{Z}}

\newcommand{\Esp}{\mathbb{E}}
\newcommand{\Prob}{\mathbb{P}}
\newcommand{\Var}{\text{Var}}

\newcommand{\A}{G}
\newcommand{\F}{\mathbb{F}}

% \newcommand{\GF}{\text{GF}}


\renewcommand{\epsilon}{\varepsilon}



\title[Attaque de Chor-Rivest]{Attaque de Sidelnikov-Shestakov appliquée au cryptosystème de Chor-Rivest }

\subtitle{INF 581 \ - \ Enseignement d'Approfondissement \\ D. Augot}

\author[S. Colin \& G. Férey]{Sylvain Colin \& Gaspard Férey}

\institute[X 2011]{Département d'Informatique\\ Ecole Polytechnique, France }
\date{16 Décembre 2013}


%\logo{\includegraphics[height=1cm]{pictures/logo.png}}
\titlegraphic{\includegraphics[height=1.5cm]{pictures/logo.png}}




\begin{document}


%------- the titlepage frame ---------------%
\begin{frame}[plain]
  \titlepage
\end{frame}


\section{Attaque de Sidelnikov-Shestakov}

\subsection{Cryptosystème de McEliece utilisant les codes de Reed-Solomon}

\begin{frame}{Cryptosystème de McEliece utilisant les codes de Reed-Solomon}

Du texte
\begin{itemize}
\item un point
\end{itemize}

\end{frame}


\subsection{Attaque de Sidelnikov et Shestakov}

\begin{frame}{Attaque de Sidelnikov et Shestakov}

\end{frame}



\section{Cryptosystème de Chor-Rivest}


\subsection{Le cryptosystème de Chor-Rivest}
\begin{frame}{Le cryptosystème de Chor-Rivest}
Clef privée:
\begin{itemize}
\item $t \in \F_q$ dont le polynôme minimal est de degré $h$.
\item $g$ générateur $\F_q^*$.
\item $0 \leq d < q$.
\item $\pi$ permutation de $\{ 0, ... , p-1 \}$.
\end{itemize}
Clef publique:
$$ c_i := d + \log_g(t + \alpha_{\pi(i)}) \mod q-1 $$
Message $m = [m_0...m_{p-1}]$ avec $\sum_i m_i = h$.
Message chiffré:
$$ E(M) := \sum_{i=0}^{p-1} m_i c_i \mod q-1 $$
On déchiffre en calculant
$$ g^{E(M) - hd} =  \prod_i \left( t + \alpha_{\pi(i)}\right)^{m_i} $$
\end{frame}




\subsection{Notre attaque}

\begin{frame}{Lien avec Reed-Solomon}
\begin{theorem}
\label{thm:link}
Pour $2 \leq k \leq p-2$, supposons qu'il existe $k$ polynômes $(Q_i)_{0 \leq i \leq k-1}$ de $\F_p[X]$ de degré inférieur à $k-1$ linéairement indépendants. Supposons connues les évaluations de ces polynômes en les $\alpha_{\pi(j)}$, $m_{i,j} := Q_i(\alpha_{\pi(j)})$.
Alors la permutation $\pi$ peut être retrouvée en temps polynomial en utilisant une attaque de Sidelnikov-Shestakov sur la matrice $ M = (m_{i,j})_{i,j} \in \mathcal{M}_{k,p}(\F_p)$.
\end{theorem}
\end{frame}


\begin{frame}{Attaque de Vaudenay}
\begin{theorem}
Quelque soit  $r$ divisant $h$, il existe un générateur $g_{p^r}$ du groupe multiplicatif $\F_{p^r}^*$ (où $F_{p^r}$ sous-corps de $\F_q$) et $Q \in \F_{p^r}[X]$ de degré $h/r$ admettant $-t$ pour racine et tel que pour tout $j$, $Q(\alpha_{\pi(j)}) = g_{p^r}^{c_j}$.
\end{theorem}
\begin{proof}
On a $g_{p^r} = g^{\frac{q-1}{p^r-1}}$ et
$$ Q(X) = g_{p^r}^d \prod_{i=0}^{h/r-1} \left( X + t^{p^{ri}} \right) $$
\end{proof}
\end{frame}


\begin{frame}{Attaque de Vaudenay}
\begin{theorem}
Si $r > \sqrt{h}$, et $g_{p^r}$ connu, il existe une attaque du cryptosystème de Chor-Rivest en temps polynomial.
\end{theorem}
\begin{proof}
Les $r$ coordonnées de $g_{p^r}^{c_j}$ sont des polynômes de degré $h/r > r$ en les $\alpha_{\pi(j)}$.
On utilise une attaque de Sidelnikov-Shestakov sur la matrice de ces coordonnées.
\end{proof}
\end{frame}


\begin{frame}{Utilisation des puissances de $g_{p^r}$}
Soit $r$ diviseur de $h$ et $(e_i)_{1 \leq i \leq r}$ une base de $\F_{p^r}$. On note
\begin{itemize}
\item $ U_w := \{ u \in [0,p^r-1] | w_p(u) \leq w \} $
\item $h[i]$ la $i$ème coordonnée de $h \in \F_{p^r}$ dans la base $(e_i)$.
\item $$ M^{(w)} := \left(g_{p^r}^{uc_j}[i] \right)_{(u,i) \in [1,r] \times U_w , 1 \leq j \leq r}$$
\item $u_w := \text{rank} \left( M^{(w)} \right)$
\end{itemize}
On a
$$ u_w \leq r \cdot |U_w| = O\left( \frac{w^{r+1}}{r!} \right) $$


\end{frame}


\subsection{Simulations et complexité espérée}
\begin{frame}{Postulat}

\begin{postulate}
Pour tout $r > 2$,
$$ u_w = \min \left( \binom{w+r}{r}, w\frac{h}{r} + 1 , p \right).$$
\end{postulate}

\end{frame}




\end{document}
